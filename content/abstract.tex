\pagenumbering{roman}
\setcounter{page}{1}

\selecthungarian

%----------------------------------------------------------------------------
% Abstract in Hungarian
%----------------------------------------------------------------------------
\chapter*{Kivonat}\addcontentsline{toc}{chapter}{Kivonat}

Az online e-learning platformok forradalmasították a tanulást, lehetővé téve a rugalmas időbeosztást, és lehetőséget adva a diákoknak, hogy a saját tempójukban haladjanak. A nagy nyelvi modellek (LLM-ek) új dimenziót nyitnak a tanulásban, interaktív és személyre szabott élményeket nyújtva. Ez a munka célja a \texttt{SpacedAce} nevű, mesterséges intelligenciával működő tanulási platform megalkotása és bemutatása, amely valóban személyre szabott és lebilincselő oktatási élményt nyújt.

A platform magja Go programozási nyelven íródott, az Echo webes keretrendszer használatával, amely robusztus és nagy teljesítményű háttérinfrastruktúrát biztosít. A rendszer felhasználói felülete a HTMX-t használja, egy kivételes frontend könyvtárat, amely szinte teljesen JavaScript-mentesen képes interaktív és dinamikus webes felületeket létrehozni.

A platform egy finomhangolt, előtanított LLM-et integrál, amely a felhasználó által megadott tananyagból kérdéseket generál. Három típusú kérdés generálását támogatja: Egy válaszlehetőséges feleletválasztós, Több válaszlehetőséges feleletválasztós, valamint Igaz/Hamis típusú kérdéseket.

A főbb funkciók közé tartozik a kérdésgenerálás, kvízek létrehozása, tananyagok ütemezése, automatikus értékelés és pontozás. A tananyagok személyre szabott, időközönkénti ismétlés-alapú algoritmus segítségével kerülnek automatikus ütemezésre, amely igazodik az egyéni tanulási igényekhez és optimalizálja a tudásmegtartást.

A felvázolt e-learning platform bemutatja a feltörekvő webtechnológiák, különösen a HTMX, innovatív alkalmazását, és szemlélteti az LLM-ek oktatástechnológiában rejlő potenciálját.



\vfill
\selectenglish


%----------------------------------------------------------------------------
% Abstract in English
%----------------------------------------------------------------------------
\chapter*{Abstract}\addcontentsline{toc}{chapter}{Abstract}

Online e-learning platforms have revolutionized learning, enabling flexible scheduling and allowing students to progress at their own pace. Large language models (LLM) take learning to a new dimension by providing interactive and personalized experiences. This work aims to create and demonstrate \texttt{SpacedAce}, an AI-powered learning platform designed to provide a truly personalized and engaging educational experience.

The platform's core is written in Go using the Echo web framework, which provides a robust and performant backend infrastructure. The system's frontend utilizes HTMX, an extraordinary frontend library that promises nearly zero JavaScript to create interactive and dynamic web interfaces.

The platform integrates a fine-tuned pre-trained LLM to generate questions from user-provided learning material. It supports generating three types of questions: Multiple Choice - Single Answer, Multiple Choice - Multiple Answers, and Boolean, also known as True or False.

Key features include question generation, quiz creation, learning material scheduling, automatic evaluation, and scoring. The learning materials are automatically scheduled using a personalized, spaced, repetition-based algorithm that adapts to individual learning needs and optimizes knowledge retention.

The proposed e-learning platform demonstrates the innovative application of emerging web technologies, specifically HTMX,  and illustrates the potential of  LLMs in educational technology.

\vfill
\selectthesislanguage

\newcounter{romanPage}
\setcounter{romanPage}{\value{page}}
\stepcounter{romanPage}