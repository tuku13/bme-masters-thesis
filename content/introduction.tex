\chapter{\bevezetes}

This chapter briefly introduces the topic, why I chose it, what my motivation was behind the problem, and how I approached solving it. I will also examine other solutions available on the market and then compare them to my own in their context. Finally, I will present the structure of my thesis in the last section.

\section{Background}

\texttt{SpacedAce} is a modern e-learning platform offering personalized learning experiences and AI-powered quiz generation. It utilizes the spaced repetition learning technique popular among flashcard applications to maximize the user's performance and minimize the time spent on learning. It also integrates a state-of-the-art AI model to help create the learning material for the users from their context. The platform is co-created with Máté Debreczeni, who made the AI model and integrated it into the platform. He authored the training process, its results, and the integration into the platform in his master's thesis \textt{Tailored Learning Experiences: Custom LLM Integration in an E-Learning Platform}. I have been working on implementing the software part of the platform, and now I describe its details in this thesis.

One of the challenges in building the platform was dealing with unique and unfamiliar technologies. I have been a software developer specializing in web development for over two years. Throughout that time, I have tried a few frameworks and tools for both work and hobby projects; most of them are well-known and used, but there were and always are alternatives that aim to revolutionize the current state of the web. I am always eager to dive in and give it a try. And that was the case with the HTMX\footnote{https://htmx.org/}, an untraditional web framework that takes a unique approach to building modern web applications. It does not tie you down but gives you the freedom and responsibility to choose the technologies involved. That was why I chose it as a first place.

\section{Motivation}

In recent years, AI technology has advanced significantly and become widely available. One of its kind, the large language model (LLM), can understand and respond to human text, opening up endless possibilities for its use. It can be utilized in education to deliver personalized education for students.

Personalized education has recently become the focus when we realize we have different needs and skills. Adjusting education to personal needs is effective, but it is expensive. Traditionally, it meant private lessons or private tutoring, but in recent years, platforms and applications specialized in personal education have appeared. These software offer different solutions, including online private lessons, personalized learning materials, and adaptive learning techniques like spaced repetition.

\subsection{Existing solutions}

There are different solutions for the case mentioned before, which have advantages and disadvantages. Here are some of the following:

\begin{itemize}
    \item{Quizlet\footnote{https://quizlet.com/}}: Quizlet is a digital, flashcard-based learning platform with AI-powered study tools. Their solutions focus on manual flashcards and question creation and have an AI tutor feature.
    \item{Revisely\footnote{https://www.revisely.com/}}: Revisely is an AI-powered learning platform focusing on AI-generated flashcards, quizzes, and notes. Their solution can create flashcards from user-provided content and practice them with the user using spaced repetition.
    \item{Questgen\footnote{https://www.questgen.ai/}}: Questgen is and AI-powered quiz generator. Their application can generate customizable questions from user content and create quizzes with different question types.
\end{itemize}

All the listed solutions above have either an AI-powered question generation or personalized learning algorithms, but none combine them as \texttt{SpacedAce} does.

\section{Overview of structure}

The thesis structure follows: Chapter ~\ref{ch:specification} specifies the task and shows the platform's use cases. Chapter~\ref{ch:theoretical-background} introduces three main concepts and the theory behind them. Chapter~\ref{ch:technologies} explains the core technologies used to build the platform. Chapter~\ref{ch:design} details the platform on the design side, while Chapter~\ref{ch:implementation} details it on the implementation side. Finally, \ref{ch:conclusion-and-future-work} summarizes the work and the results and brings up ideas for future work.
